\documentclass[10pt,twocolumn,letterpaper,preprint]{article}
\usepackage{mathtools}
\usepackage{abstract}
\usepackage{fontspec}
\usepackage{marvosym}
\setmainfont{Crimson Text}[Ligatures=TeX]
\usepackage{hyperref}
\hypersetup{
    colorlinks = true,
    allcolors = blue
}
\usepackage{url}
\begin{document}
\title{Newtonian Gravity}
\author{Robert J.\ Hansen\thanks{\href{mailto:rob@hansen.engineering}{\nolinkurl{rob@hansen.engineering}}}}
\twocolumn[
    \maketitle
    \begin{onecolabstract}
    Isaac Newton's law of universal gravitation sounds complicated, but in
    reality you can do surprising things armed with it and a little high
    school algebra.  Here we're going to use it to figure out, ``So where
    between the earth and the moon would their two gravities cancel out?'',
    and for an encore using a little calculus, ``and how much energy does it
    take to get there?''

    The latest version of this document can be found at
    \href{https://rjhansen.github.io/fun-math}{GitHub}.

    Copyright 2017, Robert J.\ Hansen.  You may share and adapt this work
    under the terms of the \href{https://creativecommons.org/licenses/by-sa/4.0/}{Creative Commons Attribution-ShareAlike
    4.0 International license}.
    \end{onecolabstract}
]
\saythanks
\tableofcontents
\section{Introduction}
Imagine you wanted to send a baseball all the way to the moon.  How far would
you have to hit it?

The moon is about 384,000 kilometers away, but you
wouldn't need to hit it that far.  Somewhere between the earth and the moon,
there's a \textit{libration point} where the earth's gravity and the moon's
gravity cancel out.  If you're even one meter short of this libration point
you'll fall back to earth --- but if you go one meter beyond, the moon's gravity
will draw you the rest of the way to the moon.

So where is the libration point?  How far would we have to hit a baseball so
the moon's gravity could take it the rest of the way?  With a little algebra
and Newton's theory of gravity, we can figure it out.

\subsection{Required skills}
In order to make sense of this you'll need to have a good handle on basic
algebra in one variable, square roots, and exponents.  This won't teach you
any algebra, but it will show you how you can use algebra to answer
interesting questions about the universe.

\section{Newton's Law of Universal Gravitation}

In 1687, Isaac Newton published a groundbreaking science book called the
\textit{Philosophiae Naturalis Principia Mathematica}.  Like all science books
of his day, he wrote it in Latin.  Its name means, ``The Mathematical
Principles of Physics''.\footnote{Back then, physics was called ``natural
philosophy''.}  Most people just call it the \textit{Principia}.

In it, Newton put forth the following idea about how gravity works.  Newton
said the force of gravity between two objects was equal to what you get
by multiplying a certain constant with the masses of the two objects, and
dividing that by the square of how far apart they are.\footnote{
In everyday use, people tend to use ``mass'' and ``weight'' as synonyms.
Physicists don't.  We use mass to mean \textit{how much there is,} and weight
to mean \textit{how hard it's being pulled down.}  If you were to travel to
the moon you'd have the same mass, but would weigh only a sixth what you do
on earth.
}

By long-standing custom, constants (things that don't change in an
equation) receive capital letters.  Variables (things which are allowed
to change) receive lower-case letters.  In Newton's idea for how gravity
works, the masses of the two things remain constant, as does the special value
they get multiplied by.  What varies is the distance between them.  Or, in
mathematical-ese:

\[
F = G\frac{M_1 M_2}{d^2}
\]

There are other ways this could be written.  You may sometimes see this as the
equivalent,

\[
F = \frac{G M_1 M_2}{d^2}
\]

They mean the same thing: ``take the gravitational constant, multiply it by
the first mass, multiply that by the second mass, and divide everything by the
square of the distance between them.''  As you can tell, $G$, $M_1$ and $M_2$
are constants, but the distance between them, $d$, is allowed to vary.

\subsection{The gravitational constant}
It took many scientists many years to figure out the value of $G$ in this
equation.  Today our best estimate is that $G = 6.67408 \times 10^{-11}$.  That
number is such a mouthful to say that most physicists will just call it $G$
and let that letter stand in for it all.

\subsection{An example}
Imagine you have a 1-kilogram brick and you want to know how much it weighs.
You'll need to know the mass of the earth, as well as how far from the earth's
center you are.  The earth's mass is $5.972 \times 10^{24}$ kilograms, and
standing on the earth you're about $6.371 \times 10^6$ meters from its center.

\[
F = \overbrace{6.67408 \times 10^{-11}}^\text{gravitational constant}
\times \frac{\overbrace{5.972 \times 10^{24}}^\text{mass of earth} \times
\overbrace{1}^\text{mass of brick}}{\underbrace{(6.371 \times 10^6)^2}_{\text{
distance from earth's center, squared}}}
\]

Start by removing the $1$ from the numerator.  Multiplying by 1 doesn't change
anything, so we can remove it without changing anything, and it makes things
a little nicer.

\[
F = \overbrace{6.67408 \times 10^{-11}}^\text{gravitational constant}
\times \frac{\overbrace{5.972 \times 10^{24}}^\text{mass of earth}}
{\underbrace{(6.371 \times 10^6)^2}_{\text{
distance from earth's center, squared}}}
\]

Now we're going to move the gravitational constant into the numerator:

\[
F = \frac{\overbrace{6.67408 \times 10^{-11}}^\text{gravitational constant}
\times \overbrace{5.972 \times 10^{24}}^\text{mass of earth}}
{\underbrace{(6.371 \times 10^6)^2}_{\text{
distance from earth's center, squared}}}
\]

Since multiplication is commutative, we can re-order multiplicands however
we like.  We do that next to get the powers next to each other.  (Since we're
breaking up the multiplicands, we're also going to stop with the helpful
brackets --- but don't worry, it's all for the best.)

\[
F = \frac{6.67408 \times 5.972 \times 10^{-11} \times 10^{24}}{(6.371 \times
10^6)^2}
\]

To multiply together two exponential numbers with the same base, just add
their exponents.  $-11 + 24$ is the same as $24 - 11$, or $13$.

\[
F = \frac{6.67408 \times 5.972 \times 10^{13}}{(6.371 \times 10^6)^2}
\]

$6.67408$ is about $7$, and $5.9$ is about $6$, so we should expect to see
a numerator of about $42 \times 10^{13}$.  Actually doing the math gives about
$39.86$, so our estimate was good.

\[
F = \frac{39.86 \times 10^{13}}{(6.371 \times 10^6)^2}
\]

When using scientific notation, if the leading term is $10$ or greater, divide
it by $10$ and add one to the exponent.  Keep doing this until the leading term
is smaller than $10$.  \mbox{$39.86 \geq 10$,} so we divide it by $10$ and add one to
the exponent.

\[
F = \frac{3.986 \times 10^{14}}{(6.371 \times 10^6)^2}
\]

With the numerator conquered, let's look at the denominator.  When we have
something like $(ab)^2$, we know that's really $a^2b^2$.  So with this,

\[
F = \frac{3.986 \times 10^{14}}{(\underbrace{6.371}_{a} \times
\underbrace{10^6}_{b})^2}
\]

becomes

\[
F = \frac{3.986 \times 10^{14}}{6.371^2 \times {10^6}^2}
\]

When raising a power to a power, the powers aren't added, they're multiplied.
So the denominator will be about $36 \times 10^{12}$, or about $3.6 \times
10^{13}$.  Actually multiplying $6.371 \times 6.371$ gives about $40.59$, so
our estimate was again good.

\[
F = \frac{3.986 \times 10^{14}}{4.059 \times 10^{13}}
\]

As it turns out, we can divide both the top and bottom by a common factor of
$10^{13}$.  This gives us,

\[
F = \frac{3.986 \times 10^1}{4.059}
\]

$10^1 = 10$.  That's a nice convenient number to move out, so let's do that:

\[
F = 10 \times \frac{3.986}{4.059}
\]

We've now reduced this fraction down so far we can just do some long division.
$3.986 \div 4.059 = 0.982$.  Multiplying that by ten gives

\[
F = 9.82
\]

We can now say, ``under Newton's theory of gravity, a one-kilogram block at
the earth's surface should be pulled down with a force of 9.82...''

\subsection{The unit of force}

Nine point eight two what?  Pounds?  No, not pounds --- pounds aren't part of
the metric system.  Kilograms?  No --- kilograms measure mass, not force.

The unit of force is named the \textit{newton}, after Isaac Newton.  Where
the pound is abbreviated ``lb.''\ and the kilogram is abbreviated ``kg'', the
newton is abbreviated with a capital letter N.

At long last we're finished:

\[
F = 9.82 N
\]

\subsection{Homework questions}
\begin{enumerate}
\item
If you wanted to find out how hard gravity would pull down a 1-kg brick on
the moon, what would you need to know about the moon before you could start
solving the problem?
\item
So how hard should a 1-kilogram brick be pulled down on the surface of the moon?
\item
Repeat it for Mars.  Which of the three (the earth, the moon, and Mars) has the
strongest gravity at their surface?
\end{enumerate}

\section{Finding a libration point}

To find a libration point we need to find where the force of earth's gravity is
exactly countered by the force of the moon's gravity.  Once more, let's break
out our trusty 1-kilogram brick (a useful companion in almost any physics
experiment) and figure out where the forces on it are equal.

To help keep things straight (``is that the force from the earth, or from the
moon?'') we'll use two special symbols: $e$ will denote the earth, and
$m$ will denote the moon.  $F_{e}$ would be, for instance, ``the force
exerted by the earth'', and $M_{m}$ would be, ``the mass of the moon''.

We'll let $D$ be the total distance between the center of the earth and the center of
the moon.  If the brick is $x$ meters from earth, it will be $D-x$ meters from
the moon.

First, we write out the formula for the force on the brick from the earth and
the moon:

\begin{align*}
F_{e} &= G\frac{M_{e} \times 1kg}{x^2}\\
F_{m} &= G\frac{M_{m} \times 1kg}{(D-x)^2}
\end{align*}

Just like before we're going to erase our multiply-by-one term.  Multiplying by
one changes nothing, so we can get rid of it.

\begin{align*}
F_{e} &= G\frac{M_{e}}{x^2}\\
F_{m} &= G\frac{M_{m}}{(D-x)^2}
\end{align*}

Now, we want to know where $F_e = F_m$.  Since we have a neat equation for each,
let's set those equations as equal.

\[
G\frac{M_e}{x^2} = G\frac{M_m}{(D-x)^2}
\]

Divide both sides by $G$, leaving:

\[
\frac{M_e}{x^2} = \frac{M_m}{(D-x)^2}
\]

Take the square root of both sides:

\[
\frac{\sqrt{M_e}}{x} = \frac{\sqrt{M_m}}{(D-x)}
\]

Multiply both sides by $(D-x)$:

\[
\frac{(D-x)\sqrt{M_e}}{x} = \sqrt{M_m}
\]

Divide both sides by $\sqrt{M_e}$:

\[
\frac{D-x}{x} = \frac{\sqrt{M_m}}{\sqrt{M_e}}
\]

Regularize the right-hand side:

\[
\frac{D-x}{x} = \sqrt{\frac{M_m}{M_e}}
\]

Separate out the left side into:

\[
\frac{D}{x} - 1 = \sqrt{\frac{M_m}{M_e}}
\]

Add 1 to both sides:

\[
\frac{D}{x} = 1 + \sqrt{\frac{M_m}{M_e}}
\]

Divide by D:

\[
\frac{1}{x} = \frac{1 + \sqrt{\frac{M_m}{M_e}}}{D}
\]

Reciprocalize:

\[
x = \frac{D}{1 + \sqrt{\frac{M_m}{M_e}}}
\]

Knowing that $D = 3.84402 \times 10^{9}$ meters, $M_e$ is about
$5.972 \times 10^{24}$ kilograms, and $M_m$ is about $7.348 \times
10^{22}$ kilograms, let's try again:

\[
x = \frac{3.84402 \times 10^9}{1 + \sqrt{\frac{7.348 \times
10^{22}}{5.972 \times 10^{24}}}}
\]

That square root term is still mighty unpleasant, so let's reduce it.
We start by canceling out a factor of $10^{22}$:

\[
x = \frac{3.84402 \times 10^9}{1 + \sqrt{\frac{7.348}{5.972 \times 10^{2}}}}
\]

And $10^2$ is just one hundred, so let's multiply it out:

\[
x = \frac{3.84402 \times 10^9}{1 + \sqrt{\frac{7.348}{597.2}}}
\]

That fraction is now small enough to punch it into a calculator.  It's just
over one hundredth: specifically, $1.2304 \times 10^{-2}$.  The square root
of that is just over a tenth: $0.111$.  This means we have:

\begin{align*}
x &= \frac{3.84402 \times 10^9}{1 + 0.111}\\
x &= \frac{3.84402 \times 10^9}{1.111}
\end{align*}

Still unpleasant.  But by moving the $10^9$ out front, we get:

\[
x = 10^9 \times \frac{3.84402}{1.111}
\]

$3.84402 \div 1.111 = 3.45996$.  Moving the decimal place right nine places
will give us the libration point's distance from earth in meters; moving it
left three places after that will convert to kilometers.

Or we could just save ourselves a step, realize that $9 - 3 = 6$, shift the
decimal place right six places, and get our answer immediately.

\[
x = 345996 \text{km}
\]

The earth and moon are separated by $384402$ kilometers, but once you travel
$345996$ kilometers --- or about nine-tenths of the way --- the moon's gravity
has you, and you're falling to the moon from that point on!

\end{document}
